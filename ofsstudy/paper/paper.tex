\documentclass[twocolumn]{article}


\begin{document}

\title{A Preliminary Study of OpenFabrics Software (OFS)}
\author{Fredrik Pe Ingebrigtsen\\
  Norwegian University of Science and Technology,\\
  \texttt{fredripi@stud.ntnu.no}}
\date{\today}
\maketitle




\begin{abstract}
OpenFrabrics Software (OFS) is an open-source middleware that provide efficient communication for parallel programs. The key service is to provide Remote Direct Memory Access (RDMA) which enables a networked computer to access the memory of other computers without involving its processor and operating system. Thus, OFS has the potential to deliver high-throughput and low-latency communication services.

The goal of this project assignment is to study OFS with the aim of identifying strengths and potential weaknesses. Concretely, the student should conduct a literature study that relate selected implementation choices in OFS to the scientific state of the art. In addition, the student should use OFS to implement one or more kernels and compare this implementation to a conventional implementation using MPI. Applications can be collected a benchmark suite such as NAS Parallel Benchmarks or the Berkley Dwarfs.
\end{abstract}

\section{Introduction}
halo
\section{Background}
halo
\section{Scheme}
halo
\section{Experiments}
halo
\section{Results}
halo
\section{Discussion}
halo
\section{Conclusion and further work}
halo

\end{document}
